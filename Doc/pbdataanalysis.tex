\section*{Introduction}

PBDataAnalysis is a C library providing structures and functions to perform various data analysis.\\ 

It implements the following algorithms:
\begin{itemize}
\item K-means clustering (random, Forgy and plusplus seeds)
\item Principal Component Analysis
\end{itemize}

It uses the \begin{ttfamily}PBErr\end{ttfamily}, \begin{ttfamily}PBMath\end{ttfamily}, \begin{ttfamily}PBJSon\end{ttfamily}, \begin{ttfamily}GSet\end{ttfamily} libraries.\\

\section{Definitions}

\subsection{K-means clustering}

The goal of the K-means clustering algorithm is to find K Voronoi cells which clusters a data set in a way that for each cells the center of this cell is the nearest possible to the average value of the input data inside this cell.\\

The K-means algorithm is as follow, where 'seed' defines the way we initialise the algorithm: 'random', 'Forgy' or 'plusplus'.\\

As an example of use, code is provided to compute the target dimensions in the config file of YoloV3 given the target bounding boxes of the training data set, using K-means clustering.\\

\begin{scriptsize}
\begin{ttfamily}
\verbatiminput{/home/bayashi/GitHub/PBDataAnalysis/Doc/kmeansclustering.txt}
\end{ttfamily}
\end{scriptsize}

\subsection{Principal Component Analysis}

PCA consists of projecting the data on a subset of the Eigen vectors of the covariance matrix for these data.\\

\begin{scriptsize}
\begin{ttfamily}
\verbatiminput{/home/bayashi/GitHub/PBDataAnalysis/Doc/pca.txt}
\end{ttfamily}
\end{scriptsize}

\section{Interface}

\begin{scriptsize}
\begin{ttfamily}
\verbatiminput{/home/bayashi/GitHub/PBDataAnalysis/pbdataanalysis.h}
\end{ttfamily}
\end{scriptsize}

\section{Code}

\subsection{pbdataanalysis.c}

\begin{scriptsize}
\begin{ttfamily}
\verbatiminput{/home/bayashi/GitHub/PBDataAnalysis/pbdataanalysis.c}
\end{ttfamily}
\end{scriptsize}

\section{Makefile}

\begin{scriptsize}
\begin{ttfamily}
\verbatiminput{/home/bayashi/GitHub/PBDataAnalysis/Makefile}
\end{ttfamily}
\end{scriptsize}

\section{Unit tests}

\begin{scriptsize}
\begin{ttfamily}
\verbatiminput{/home/bayashi/GitHub/PBDataAnalysis/main.c}
\end{ttfamily}
\end{scriptsize}

\section{Unit tests output}

\begin{scriptsize}
\begin{ttfamily}
\verbatiminput{/home/bayashi/GitHub/PBDataAnalysis/unitTestRef.txt}
\end{ttfamily}
\end{scriptsize}

kmeancluster.csv:\\
\begin{scriptsize}
\begin{ttfamily}
\verbatiminput{/home/bayashi/GitHub/PBDataAnalysis/kmeancluster.csv}
\end{ttfamily}
\end{scriptsize}

random seed in grey, Forgy seed in black, plusplus in dark grey:\\
\begin{center}
\begin{figure}[H]
\centering\includegraphics[width=6cm]{./kmeansclustering.png}\\
\end{figure}
\end{center}

kmeancluster.txt:\\
\begin{scriptsize}
\begin{ttfamily}
\verbatiminput{/home/bayashi/GitHub/PBDataAnalysis/kmeancluster.txt}
\end{ttfamily}
\end{scriptsize}

unitTestPCA.json:\\
\begin{scriptsize}
\begin{ttfamily}
\verbatiminput{/home/bayashi/GitHub/PBDataAnalysis/unitTestPCA.json}
\end{ttfamily}
\end{scriptsize}

PCA1D:\\
\begin{center}
\begin{figure}[H]
\centering\includegraphics[width=10cm]{../PCA1D.png}\\
\end{figure}
\end{center}

PCA2D:\\
\begin{center}
\begin{figure}[H]
\centering\includegraphics[width=10cm]{../PCA2D.png}\\
\end{figure}
\end{center}

\section{YoloBoxes}

main.c:\\
\begin{scriptsize}
\begin{ttfamily}
\verbatiminput{/home/bayashi/GitHub/PBDataAnalysis/YoloBoxes/main.c}
\end{ttfamily}
\end{scriptsize}

example of use and output:\\
\begin{scriptsize}
\begin{ttfamily}
\verbatiminput{/home/bayashi/GitHub/PBDataAnalysis/YoloBoxes/res.txt}
\end{ttfamily}
\end{scriptsize}
